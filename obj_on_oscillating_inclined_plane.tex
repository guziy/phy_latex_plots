\documentclass[tikz,border=10pt]{standalone}
\usepackage{tikz}
\usetikzlibrary{scopes}
\usepackage{verbatim}
\usetikzlibrary{calc,angles,patterns,quotes}


\begin{comment}

:Title: Free body diagrams
:Slug: free-body-diagrams
:Tags: Matrices

Illustration of the classical "bodies connected by a string" problem.
A great advantage of using TikZ for drawing illustrations like this, is that the
drawings can be parameterized. In this example the inclination angle is
parameterized. PGF's mathematical engine is then used to do the necessary calculations.

Note that this example uses the shorthand scope notation in some places.

\end{comment}


\tikzset{
    right angle quadrant/.code={
        \pgfmathsetmacro\quadranta{{1,1,-1,-1}[#1-1]}     % Arrays for selecting quadrant
        \pgfmathsetmacro\quadrantb{{1,-1,-1,1}[#1-1]}},
    right angle quadrant=1, % Make sure it is set, even if not called explicitly
    right angle length/.code={\def\rightanglelength{#1}},   % Length of symbol
    right angle length=1ex, % Make sure it is set...
    right angle symbol/.style n args={3}{
        insert path={
            let \p0 = ($(#1)!(#3)!(#2)$) in     % Intersection
                let \p1 = ($(\p0)!\quadranta*\rightanglelength!(#3)$), % Point on base line
                \p2 = ($(\p0)!\quadrantb*\rightanglelength!(#2)$) in % Point on perpendicular line
                let \p3 = ($(\p1)+(\p2)-(\p0)$) in  % Corner point of symbol
            (\p1) -- (\p3) -- (\p2)
        }
    }
}



\begin{document}

\def\iangle{35} % Angle of the inclined plane
\def\down{-90}
\def\arcr{0.5cm} % Radius of the arc used to indicate angles
\def\ang{30}

\begin{tikzpicture}[
    force/.style={>=latex,draw=blue,fill=blue},
    axis/.style={densely dashed,gray},
    M/.style={rectangle,draw,fill=lightgray,minimum size=0.5cm,thin},
    m/.style={rectangle,draw=black,fill=lightgray,minimum size=0.3cm,thin},
    plane/.style={draw=black,fill=blue!10},
    string/.style={draw=red, thick},
    pulley/.style={thick}
]


    \node (rotation_center) at (0, 0) {};
    \node (angle_beg) at (1cm, 0) {};
    \node (angle_end) at (0.866cm, 0.5cm) {};

    \begin{scope}[rotate=\ang]

      \draw (0, 0) rectangle ++(4cm, 0.1cm) node (plane) {};

      % object and forces acting on it
      \draw (2.5cm, 0.1cm) rectangle ++(0.5cm, 0.25cm);
      \draw[fill=black] (2.75cm, 0.225cm) circle [radius=0.03]  node (obj_center) {}; %

      \draw[force,>=latex,->] (obj_center.center) -- ++(0, 1cm) node[right] {$\vec{N}$};
      \draw[force,>=latex,->,rotate=-\ang] (obj_center.center) -- ++(0, -1cm) node[right] {$m\vec{g}$};
      \draw[force,>=latex,->] (obj_center.center) -- ++(1cm, 0) node[above] {$\vec{F}_{i}$};
      \draw[force,>=latex,->] (obj_center.center) -- ++(-1cm, 0) node[above=0.1cm] {$\vec{F}_{f}$};

      \draw[axis,>=latex,->] (3.5cm, 1.5cm) -- ++(2cm, 0cm) node[above right] {$x$};

      \draw[force,>=latex,->] (4.5cm, 0cm) -- ++(2cm, 0) node[below, right] {$+v_0$};
      \draw[force,>=latex,->] (-0.5cm, 0cm) -- ++(-2cm, 0) node[below, left] {$-v_0$};

    \end{scope}

    % draw the angle of the inclined plane
    \draw[axis] (0, 0) -- ++(3cm, 0cm) {};
    \pic [draw, ->, "$\alpha$", angle eccentricity=1.3]{
        angle = angle_beg--rotation_center--angle_end
    };


\end{tikzpicture}



\end{document}
