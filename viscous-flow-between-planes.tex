\documentclass[tikz,border=10pt]{standalone}
\usepackage{tikz}
\usetikzlibrary{scopes}
\usepackage{verbatim}
\usetikzlibrary{calc,angles,patterns,quotes,positioning}
\usetikzlibrary{patterns.meta}
\usepackage{pgfplots}
\usepackage{pgf}
\usepackage[dvipsnames]{xcolor}
\usetikzlibrary{math}
\usetikzlibrary{backgrounds}
\pgfplotsset{compat=newest}
\pgfplotsset{plot coordinates/math parser=false}



\begin{document}

\def\down{-90}
\def\ang{30}
\def\hgt{10em}
\def\lwdth{0.2em}
\def\R0{10.7em}
\def\r0{2.7em}
\def\dotr{0.1em}
\def\parcelangle{30}

\def\totalWaterLengthX{45em}
\def\totalWaterHeightZ{21em}
\def\wallWidth{1em}
\def\parcelHalfHeight{0.2*\totalWaterHeightZ}


\begin{tikzpicture}[
    force/.style={>=latex,draw=blue,fill=blue,line width=\lwdth},
    axis/.style={>=latex,dashed,gray, line width=0.15em, 
                 dash pattern=on 8pt off 3pt},
    M/.style={rectangle,draw,fill=lightgray,minimum size=0.5cm,thin},
    m/.style={rectangle,draw=black,fill=lightgray,minimum size=0.3cm,thin},
    plane/.style={draw=black,fill=blue!10},
    string/.style={draw=red, thick},
    pulley/.style={thick},
    point/.style={inner sep=0, minimum width=0, text=black},
    timber-fill/.style={pattern={
        Lines[distance=8.5pt, angle=45]},
        pattern tile/.style={clip=false},
        pattern color=black
    },
    timber-fill-top/.style={
        pattern={Lines[distance=8.5pt, angle=135]}, 
        pattern tile/.style={clip=false},
        pattern color=black
    },
    scale=1.2
]


    \tikzstyle{every node}=[font=\Large];

    \node[point] at (0,0) (lowerPlateLeftEdge) {};

    \fill[Aquamarine!45] (lowerPlateLeftEdge.center) -- ++(\totalWaterLengthX, 0) 
                                    -- ++(0, \totalWaterHeightZ) node[point, midway] (middleWaterLayerRight) {}
                                    -- ++(-\totalWaterLengthX, 0) node[point] (upperPlateLeftEdge) {}
                                    -- cycle node[point, midway] (middleWaterLayerLeft){};

    \draw[line width=0.1em] (upperPlateLeftEdge.center) -- ++(\totalWaterLengthX, 0) 
                                        node[point] (upperPlateRightEdge) {};
    
    \draw[line width=0.1em] (lowerPlateLeftEdge.center) -- ++(\totalWaterLengthX, 0) 
                                        node[point] (lowerPlateRightEdge) {}; 
                                        
                                        
    \draw[axis, ->] (lowerPlateLeftEdge.center) -- 
                  ++(1.2*\totalWaterLengthX, 0) node[below=1ex, point] (xAxisLabel){$x$};
    
    \draw[axis, ->] (lowerPlateLeftEdge.center) -- 
                  ++(0, 1.4*\totalWaterHeightZ) node[left=1ex, point] (yAxisLabel){$z$};


    \node[below left=1ex, point] at (lowerPlateLeftEdge) {$O$};


    \pattern[timber-fill, draw=none] (lowerPlateLeftEdge.center) -- ([xshift=-0.1em]lowerPlateRightEdge.center) 
                   -- ++(0, -\wallWidth) -- ++(-\totalWaterLengthX, 0) --cycle;

    \pattern[timber-fill-top, draw=none] (upperPlateLeftEdge.center) -- (upperPlateRightEdge.center) 
                   -- ++(0, \wallWidth) -- ++(-\totalWaterLengthX, 0) --cycle;

    \draw[axis] (middleWaterLayerLeft.center) -- (middleWaterLayerRight.center) 
                node[pos=0.3, point] (leftMiddleParcel) {}
                node[pos=0.7, point] (rightMiddleParcel) {}
                node[pos=0.5, point] (centerParcel) {}; 


    \draw[force, ->] (middleWaterLayerRight.center) -- ++(3em, 0) node[above right] {$v = v(h/2)$};

    % draw the imaginary parcel
    \path (leftMiddleParcel.center) -- ++(0, \parcelHalfHeight) node[point](parcelUpperLeft) {};
    \path (leftMiddleParcel.center) -- ++(0, -\parcelHalfHeight) node[point](parcelLowerLeft) {};

    \path (rightMiddleParcel.center) -- ++(0, \parcelHalfHeight) node[point](parcelUpperRight) {};
    \path (rightMiddleParcel.center) -- ++(0, -\parcelHalfHeight) node[point](parcelLowerRight) {};

    \draw[axis, violet, line width=0.2em] (parcelLowerLeft.center) 
                     -- (parcelUpperLeft.center)
                     -- (parcelUpperRight.center) node[point, midway] (upperStressStart) {}
                     -- (parcelLowerRight.center) -- cycle node[point, midway] (lowerStressStart) {};

    \draw[axis] (parcelUpperRight.center) -- ++(2.5em, 0) node [point, midway] (dzUpperLimit){};
    \draw[axis] (parcelLowerRight.center) -- ++(2.5em, 0) node [point, midway] (dzLowerLimit){};                 
    \draw[axis, <->] (dzLowerLimit.center) -- (dzUpperLimit.center) node[pos=0.7, right=1ex, point, anchor=west] {$\Delta z$};
    \draw[axis, <-] (dzLowerLimit.center) node [point, below right=2ex, anchor=north west]{$z$} --
                    ($(lowerPlateLeftEdge.center)!(dzLowerLimit.center)!(lowerPlateRightEdge.center)$);

                      
    % mark forces acting on the parcel
    \draw[force, ->] (upperStressStart.center) -- ++(-6em, 0) node[above=1ex, anchor=south, point] {$\tau_x(h - z)=\tau_x(z)$};
    \draw[force, ->] (lowerStressStart.center) -- ++(-6em, 0) node[below=1ex, anchor=north, point] {$\tau_x(z)$};


    % mark distance between horizontal plates
    \draw[axis] (lowerPlateLeftEdge.center) -- ++(-3em, 0) node[point, midway] (topHMark) {};
    \draw[axis] (upperPlateLeftEdge.center) -- ++(-3em, 0) node[point, midway] (botHMark) {};
    \draw[axis, <->] (topHMark.center) -- (botHMark) node [point, midway, left=2ex] {$h$}; 

    


    
\end{tikzpicture}

\end{document}